\chapter{Interfaccia Utente}

\section{Design dell'Interfaccia}

L'interfaccia utente è stata progettata seguendo principi di user experience per rendere 
accessibili operazioni complesse anche a utenti non tecnici. Il design si basa su:

\begin{itemize}
    \item \textbf{Chiarezza}: Ogni operazione è guidata con istruzioni e feedback
    \item \textbf{Semplicità}: Complessità nascosta dietro preset e configurazioni automatiche
    \item \textbf{Immediatezza}: Feedback visivo in tempo reale
    \item \textbf{Coerenza}: Pattern di interazione uniformi in tutta l'applicazione
\end{itemize}

\section{Struttura dell'Interfaccia}

\subsection{Header e Branding}

L'header principale presenta il titolo con effetto gradient e glow semi-fluorescente:

\begin{lstlisting}[caption={Header con gradient CSS}]
st.markdown("""
    <div style='text-align: center; padding: 1rem 0 2rem 0;'>
        <h1 style='margin: 0; font-size: 4.5rem; font-weight: 700; 
                   background: linear-gradient(135deg, #667eea 0%, #764ba2 100%);
                   -webkit-background-clip: text;
                   -webkit-text-fill-color: transparent;
                   text-shadow: 0 0 30px rgba(102, 126, 234, 0.3),
                                0 0 60px rgba(118, 75, 162, 0.2);'>
            Steganography WebApp
        </h1>
        <p style='font-size: 1.1rem; opacity: 0.7;'>
            Hide data within images using advanced algorithms
        </p>
    </div>
    """, unsafe_allow_html=True)
\end{lstlisting}

\subsection{Sidebar - Selezione Metodo}

La sidebar contiene cards cliccabili per selezionare l'algoritmo:

\begin{lstlisting}[caption={Sidebar con cards metodi}]
with st.sidebar:
    st.markdown("### Metodo Steganografico")
    
    # Card LSB
    if st.button("LSB\nVeloce - Alta capacita", 
                 use_container_width=True):
        st.session_state.selected_method = SteganographyMethod.LSB
        st.rerun()
    
    # Card DWT
    if st.button("DWT\nRobusto - Qualita", 
                 use_container_width=True):
        st.session_state.selected_method = SteganographyMethod.DWT
        st.rerun()
    
    # Card PVD
    if st.button("PVD\nAdattivo - Versatile", 
                 use_container_width=True):
        st.session_state.selected_method = SteganographyMethod.PVD
        st.rerun()
\end{lstlisting}

\subsection{Selezione Tipo di Dato}

Cards orizzontali per selezionare il tipo di dato da nascondere:

\begin{lstlisting}[caption={Data type selector con columns}]
col1, col2, col3 = st.columns(3)

with col1:
    if st.button("Testo", use_container_width=True):
        st.session_state.selected_data_type = "Stringhe"

with col2:
    if st.button("Immagini", use_container_width=True):
        st.session_state.selected_data_type = "Immagini"

with col3:
    if st.button("File Binari", use_container_width=True):
        st.session_state.selected_data_type = "File binari"
\end{lstlisting}

\section{Workflow Utente}

\subsection{Operazione Hide (Nascondere)}

Il workflow per nascondere dati è strutturato in passi sequenziali:

\begin{enumerate}
    \item \textbf{Selezione metodo} (sidebar): LSB / DWT / PVD
    \item \textbf{Selezione tipo dato}: Testo / Immagini / File binari
    \item \textbf{Upload immagine host}: File uploader con anteprima
    \item \textbf{Input dati da nascondere}: 
          \begin{itemize}
              \item Testo: Text area
              \item Immagine: File uploader
              \item File binario: File uploader + selezione compressione
          \end{itemize}
    \item \textbf{Configurazione parametri}: Preset o personalizzati
    \item \textbf{Esecuzione}: Pulsante "Nascondi" con spinner
    \item \textbf{Risultati}: Anteprima + metriche + download
\end{enumerate}

\subsection{Operazione Recover (Recuperare)}

Il workflow di recupero è più semplice:

\begin{enumerate}
    \item \textbf{Selezione metodo} (sidebar)
    \item \textbf{Selezione tipo dato}
    \item \textbf{Upload immagine con dati nascosti}
    \item \textbf{Modalità recupero parametri}:
          \begin{itemize}
              \item Backup file (automatico)
              \item Backup recente
              \item Parametri manuali
          \end{itemize}
    \item \textbf{Esecuzione}: Pulsante "Recupera" con spinner
    \item \textbf{Risultati}: Dati recuperati + download
\end{enumerate}

\section{Componenti UI Riutilizzabili}

\subsection{File Upload con Anteprima}

\begin{lstlisting}[caption={Component per upload immagini}]
class ImageDisplay:
    @staticmethod
    def show_resized_image(uploaded_file, caption, max_width=400):
        """Mostra immagine caricata con resize"""
        img = Image.open(uploaded_file)
        st.image(img, caption=caption, width=max_width)
    
    @staticmethod
    def show_image_details(uploaded_file, title):
        """Mostra dettagli tecnici immagine"""
        img = Image.open(uploaded_file)
        
        with st.expander(title):
            col1, col2 = st.columns(2)
            with col1:
                st.write(f"**Dimensioni:** {img.width} x {img.height}")
                st.write(f"**Formato:** {img.format}")
            with col2:
                st.write(f"**Modalita:** {img.mode}")
                size_kb = len(uploaded_file.getvalue()) / 1024
                st.write(f"**Dimensione:** {size_kb:.2f} KB")
\end{lstlisting}

\subsection{Download Button con Icone}

\begin{lstlisting}[caption={Download button personalizzato}]
def create_download_button(data, filename, mime, label):
    """Crea pulsante download con icona"""
    st.download_button(
        label=label,
        data=data,
        file_name=filename,
        mime=mime,
        use_container_width=True,
        type="primary"
    )
\end{lstlisting}

\subsection{Preset Configurabili}

Ogni algoritmo offre preset ottimizzati:

\begin{lstlisting}[caption={Preset selector con descrizioni}]
preset = st.selectbox(
    "Preconfigurazione:",
    options=[
        "Bilanciato",
        "Alta Qualita",
        "Alta Capacita",
        "Personalizzato"
    ],
    help="Bilanciato: ottimo per uso generale. "
         "Alta Qualita: minime modifiche visibili. "
         "Alta Capacita: massimizza i dati nascosti."
)

if preset == "Bilanciato":
    n, div = 4, 0.0
    st.info("N=4, DIV=auto - Buon compromesso")
elif preset == "Alta Qualita":
    n, div = 1, 0.0
    st.info("N=1, DIV=auto - Massima qualita visiva")
# ...
\end{lstlisting}

\section{Feedback Visivo e User Experience}

\subsection{Indicatori di Progresso}

Durante operazioni lunghe, viene mostrato uno spinner:

\begin{lstlisting}[caption={Spinner con messaggio}]
with st.spinner("Nascondendo messaggio con DWT..."):
    result_img, metrics = hide_message(img, message)
    
st.success("Messaggio nascosto con successo!")
\end{lstlisting}

\subsection{Visualizzazione Metriche}

Le metriche di qualità sono presentate con st.metric:

\begin{lstlisting}[caption={Metriche con delta}]
col1, col2 = st.columns(2)

with col1:
    st.metric(
        label="SSIM (Similarita Strutturale)",
        value=f"{metrics['ssim']:.4f}",
        help="1.0 = immagini identiche"
    )

with col2:
    st.metric(
        label="PSNR (Rapporto Segnale/Rumore)",
        value=f"{metrics['psnr']:.2f} dB",
        help="Valori piu alti = migliore qualita"
    )
\end{lstlisting}

\subsection{Messaggi di Errore Informativi}

Gli errori sono presentati con contesto utile:

\begin{lstlisting}[caption={Gestione errori user-friendly}]
try:
    result = hide_message(img, msg)
except ValueError as e:
    st.error(f"""
        **Errore di validazione**
        
        {str(e)}
        
        **Suggerimento:** Prova a:
        - Usare un'immagine piu grande
        - Ridurre la lunghezza del messaggio
        - Comprimere i dati (per file binari)
    """)
\end{lstlisting}

\section{Responsività e Accessibilità}

L'interfaccia si adatta a diverse dimensioni di schermo grazie al layout responsive di 
Streamlit:

\begin{itemize}
    \item \textbf{Desktop}: Layout a colonne con sidebar
    \item \textbf{Tablet}: Colonne ridotte, sidebar collassabile
    \item \textbf{Mobile}: Layout verticale singola colonna
\end{itemize}

\subsection{Stili CSS Personalizzati}

\begin{lstlisting}[caption={Custom CSS per tema scuro}]
st.markdown("""
    <style>
    /* Card metodi nella sidebar */
    .method-card-container {
        border-radius: 8px;
        padding: 1rem;
        margin: 0.5rem 0;
        transition: all 0.3s;
    }
    
    .method-card-container.selected {
        background: linear-gradient(135deg, #667eea20, #764ba220);
        border: 2px solid #667eea;
    }
    
    /* Footer */
    .app-footer {
        text-align: center;
        padding: 2rem 0;
        border-top: 1px solid rgba(255,255,255,0.1);
        margin-top: 3rem;
    }
    </style>
    """, unsafe_allow_html=True)
\end{lstlisting}

\section{Gestione dello Stato tra Pagine}

Il sistema mantiene lo stato tra le interazioni usando Session State:

\begin{lstlisting}[caption={Persistenza risultati}]
# Salvataggio risultati
st.session_state["hide_image_results"] = {
    "image": img_buffer.getvalue(),
    "filename": output_filename,
    "preview_image": result_img,
    "metrics": metrics,
    "backup": backup_data
}

# Visualizzazione persistente
if "hide_image_results" in st.session_state:
    results = st.session_state["hide_image_results"]
    
    st.image(results["preview_image"], caption="Risultato")
    
    # Download disponibile fino a refresh
    create_download_button(
        results["image"],
        results["filename"],
        "image/png",
        "Scarica immagine"
    )
\end{lstlisting}

\newpage
