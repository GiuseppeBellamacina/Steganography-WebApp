\chapter{Fondamenti di Steganografia}

\section{Definizione e Storia}

La steganografia deriva dal greco \textit{steganos} (coperto) e \textit{graphein} (scrittura),
letteralmente "scrittura nascosta". Le sue origini risalgono all'antica Grecia, dove si
utilizzavano tecniche ingegnose quali tatuaggi sul cuoio capelluto rasato di messaggeri,
inchiostri invisibili e messaggi nascosti in tavole di cera. Con l'avvento dell'era digitale,
la steganografia ha trovato nuove applicazioni nel dominio dei file multimediali, trasformandosi
da arte artigianale a scienza computazionale precisa~\cite{katzenbeisser2000information}.

\section{Steganografia vs Crittografia}

La distinzione tra steganografia e crittografia è fondamentale per comprendere i diversi
approcci alla sicurezza dell'informazione. Mentre la crittografia si concentra nel rendere
illeggibile un messaggio, attirando inevitabilmente l'attenzione sulla comunicazione, la
steganografia mira a nasconderne completamente l'esistenza. Questo approccio discreto comporta
alcuni compromessi: la capacità di dati che può essere nascosta è tipicamente limitata dalla
dimensione del contenitore, e la robustezza alle manipolazioni varia significativamente a
seconda della tecnica impiegata. Tuttavia, il vantaggio principale risiede proprio nel basso
livello di sospetto generato: un'immagine contenente dati nascosti appare del tutto ordinaria
a un osservatore~\cite{provos2003hide}.

Le due tecniche possono essere efficacemente combinate, cifrando prima il messaggio e poi
nascondendolo, ottenendo così un doppio livello di sicurezza che protegge sia il contenuto
che l'esistenza stessa della comunicazione.

\section{Requisiti di un Sistema Steganografico}

Un sistema steganografico efficace deve soddisfare tre requisiti fondamentali:

\subsection{Impercettibilità (Imperceptibility)}

Le modifiche apportate al contenitore devono rimanere impercettibili all'osservazione umana.
La valutazione quantitativa di questo requisito si affida a metriche consolidate nel campo
dell'elaborazione delle immagini. Il PSNR (Peak Signal-to-Noise Ratio) misura il rapporto tra
il segnale massimo possibile e il rumore di distorsione; valori superiori a 30 dB indicano
differenze generalmente impercettibili:

$$PSNR = 10 \cdot \log_{10}\left(\frac{MAX^2}{MSE}\right)$$

dove $MAX$ rappresenta il valore massimo del pixel (255 per immagini a 8 bit) e $MSE$ l'errore
quadratico medio tra i pixel originali e modificati. Complementare al PSNR, l'indice SSIM
(Structural Similarity Index) valuta la similarità strutturale percettiva tra le immagini su
una scala da -1 a 1, dove 1 indica identità perfetta.

\subsection{Capacità (Capacity)}

La quantità di informazioni che possono essere nascoste nell'immagine host. Si misura in:
\begin{itemize}
    \item \textbf{Bit per pixel (bpp)}: Rapporto tra bit nascosti e pixel totali
    \item \textbf{Percentuale di utilizzo}: Frazione dell'immagine modificata
\end{itemize}

Esiste sempre un trade-off tra capacità e impercettibilità: aumentare la quantità di dati 
nascosti aumenta il rischio di degradazione visibile.

\subsection{Robustezza (Robustness)}

La capacità del messaggio nascosto di resistere a manipolazioni come:
\begin{itemize}
    \item Compressione JPEG
    \item Ridimensionamento
    \item Rotazione e ritaglio
    \item Aggiunta di rumore
    \item Filtri di elaborazione
\end{itemize}

Algoritmi diversi offrono diversi livelli di robustezza a seconda del dominio di embedding 
utilizzato (spaziale vs frequenza).

\section{Classificazione degli Algoritmi Steganografici}

Gli algoritmi steganografici per immagini si possono classificare in due categorie principali:

\subsection{Algoritmi nel Dominio Spaziale}

Modificano direttamente i valori dei pixel. Sono generalmente:
\begin{itemize}
    \item Veloci da implementare
    \item Ad alta capacità
    \item Poco robusti a manipolazioni
    \item Esempi: LSB, PVD
\end{itemize}

\subsection{Algoritmi nel Dominio della Frequenza}

Operano sui coefficienti di trasformazioni matematiche (DCT, DWT, DFT). Sono generalmente:
\begin{itemize}
    \item Più complessi da implementare
    \item A capacità inferiore
    \item Più robusti a compressione e filtri
    \item Esempi: DWT, DCT-based
\end{itemize}
