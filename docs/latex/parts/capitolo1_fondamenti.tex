\chapter{Fondamenti di Steganografia}

\section{Definizione e Storia}

La steganografia deriva dal greco \textit{steganos} (coperto) e \textit{graphein} (scrittura), 
letteralmente "scrittura nascosta". Le sue origini risalgono all'antica Grecia, dove venivano 
utilizzate tecniche come:

\begin{itemize}
    \item Tatuaggi sul cuoio capelluto rasato di messaggeri
    \item Inchiostri invisibili
    \item Messaggi nascosti in tavole di cera
\end{itemize}

Con l'avvento dell'era digitale, la steganografia ha trovato nuove applicazioni nel dominio 
dei file multimediali, in particolare immagini, audio e video~\cite{katzenbeisser2000information}.

\section{Steganografia vs Crittografia}

È importante distinguere tra steganografia e crittografia:

\begin{table}[H]
\centering
\begin{tabular}{|l|l|l|}
\hline
\textbf{Aspetto} & \textbf{Steganografia} & \textbf{Crittografia} \\ \hline
Obiettivo & Nascondere l'esistenza & Rendere illeggibile \\ \hline
Sospetto & Basso & Alto \\ \hline
Capacità & Limitata & Illimitata \\ \hline
Robustezza & Variabile & Molto alta \\ \hline
Chiave & Spesso opzionale & Sempre richiesta \\ \hline
\end{tabular}
\caption{Confronto tra steganografia e crittografia}
\end{table}

Le due tecniche possono essere combinate: prima cifrare il messaggio (crittografia) e poi 
nasconderlo (steganografia), ottenendo un doppio livello di sicurezza~\cite{provos2003hide}.

\section{Requisiti di un Sistema Steganografico}

Un sistema steganografico efficace deve soddisfare tre requisiti fondamentali:

\subsection{Impercettibilità (Imperceptibility)}

Le modifiche apportate al contenitore (cover) non devono essere percepibili dall'occhio umano. 
Questo si misura attraverso metriche quantitative come:

\begin{itemize}
    \item \textbf{PSNR (Peak Signal-to-Noise Ratio)}: Misura il rapporto tra il segnale 
          massimo possibile e il rumore di distorsione. Valori superiori a 30 dB indicano 
          differenze impercettibili.
          
    $$PSNR = 10 \cdot \log_{10}\left(\frac{MAX^2}{MSE}\right)$$
    
    dove $MAX$ è il valore massimo del pixel (255 per immagini a 8 bit) e $MSE$ è l'errore 
    quadratico medio.
    
    \item \textbf{MSE (Mean Squared Error)}: Calcola la media delle differenze al quadrato 
          tra pixel originali e modificati.
          
    $$MSE = \frac{1}{mn}\sum_{i=0}^{m-1}\sum_{j=0}^{n-1}[I(i,j) - K(i,j)]^2$$
    
    \item \textbf{SSIM (Structural Similarity Index)}: Valuta la similarità strutturale 
          percettiva tra le immagini (valore tra -1 e 1, dove 1 indica identità perfetta).
\end{itemize}

\subsection{Capacità (Capacity)}

La quantità di informazioni che possono essere nascoste nell'immagine host. Si misura in:
\begin{itemize}
    \item \textbf{Bit per pixel (bpp)}: Rapporto tra bit nascosti e pixel totali
    \item \textbf{Percentuale di utilizzo}: Frazione dell'immagine modificata
\end{itemize}

Esiste sempre un trade-off tra capacità e impercettibilità: aumentare la quantità di dati 
nascosti aumenta il rischio di degradazione visibile.

\subsection{Robustezza (Robustness)}

La capacità del messaggio nascosto di resistere a manipolazioni come:
\begin{itemize}
    \item Compressione JPEG
    \item Ridimensionamento
    \item Rotazione e ritaglio
    \item Aggiunta di rumore
    \item Filtri di elaborazione
\end{itemize}

Algoritmi diversi offrono diversi livelli di robustezza a seconda del dominio di embedding 
utilizzato (spaziale vs frequenza).

\section{Classificazione degli Algoritmi Steganografici}

Gli algoritmi steganografici per immagini si possono classificare in due categorie principali:

\subsection{Algoritmi nel Dominio Spaziale}

Modificano direttamente i valori dei pixel. Sono generalmente:
\begin{itemize}
    \item Veloci da implementare
    \item Ad alta capacità
    \item Poco robusti a manipolazioni
    \item Esempi: LSB, PVD
\end{itemize}

\subsection{Algoritmi nel Dominio della Frequenza}

Operano sui coefficienti di trasformazioni matematiche (DCT, DWT, DFT). Sono generalmente:
\begin{itemize}
    \item Più complessi da implementare
    \item A capacità inferiore
    \item Più robusti a compressione e filtri
    \item Esempi: DWT, DCT-based
\end{itemize}

\newpage
