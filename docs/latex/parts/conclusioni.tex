\chapter{Conclusioni}

\section{Riepilogo del Progetto}

Il progetto Steganography WebApp ha raggiunto gli obiettivi prefissati, realizzando 
un'applicazione web completa e funzionale per la steganografia digitale~\cite{cox2007digital}. L'implementazione 
di tre algoritmi distinti (LSB, DWT e PVD) offre agli utenti la flessibilità di scegliere 
l'approccio più adatto alle loro esigenze specifiche.

\subsection{Obiettivi Raggiunti}

\begin{itemize}
    \item \textbf{Implementazione multi-algoritmo}: Tre algoritmi completamente funzionanti 
          con caratteristiche complementari
    \item \textbf{Supporto multi-formato}: Gestione di stringhe, immagini e file binari 
          arbitrari
    \item \textbf{Interfaccia user-friendly}: UI intuitiva accessibile anche a utenti non 
          tecnici
    \item \textbf{Sistema di metriche}: Valutazione quantitativa della qualità (PSNR, MSE, 
          SSIM)
    \item \textbf{Automazione}: Backup parametri automatico e preset ottimizzati
    \item \textbf{Deployment online}: Applicazione accessibile su Streamlit Cloud
\end{itemize}

\section{Confronto Prestazioni}

\subsection{Risultati Sperimentali}

Test condotti su diverse immagini hanno confermato le aspettative teoriche:

\begin{table}[H]
\centering
\begin{tabular}{|l|c|c|c|}
\hline
\textbf{Metrica} & \textbf{LSB} & \textbf{DWT} & \textbf{PVD} \\ \hline
PSNR medio & 52.3 dB & 38.7 dB & 47.1 dB \\ \hline
SSIM medio & 0.9998 & 0.9945 & 0.9987 \\ \hline
Capacità (bpp) & 3.0 & 0.8 & 2.5 \\ \hline
Tempo medio* & 0.12s & 1.85s & 0.45s \\ \hline
\end{tabular}
\caption{Prestazioni medie su immagini 800×600 (*su CPU Intel i7)}
\end{table}

\subsection{Trade-off Osservati}

\textbf{LSB:}
\begin{itemize}
    \item[+] Qualità eccellente e velocità massima
    \item[-] Nessuna robustezza a compressione JPEG
\end{itemize}

\textbf{DWT:}
\begin{itemize}
    \item[+] Unico robusto a compressione JPEG (fino a 90\%)
    \item[-] Capacità limitata e computazionalmente costoso
\end{itemize}

\textbf{PVD:}
\begin{itemize}
    \item[+] Ottimo bilanciamento capacità/qualità
    \item[-] Complessità intermedia, sensibile a ridimensionamento
\end{itemize}

\section{Contributi Originali}

Rispetto alle implementazioni esistenti, il progetto introduce:

\begin{enumerate}
    \item \textbf{Header robusto multi-layer}:
          \begin{itemize}
              \item Magic header per identificazione
              \item Lunghezza messaggio per lettura precisa
              \item Checksum per verifica integrità
              \item Terminatore per sicurezza
          \end{itemize}
    
    \item \textbf{Sistema preset intelligenti}:
          \begin{itemize}
              \item Configurazioni ottimizzate per casi d'uso comuni
              \item Parametri calibrati sperimentalmente
              \item Bilanciamento automatico capacità/qualità
          \end{itemize}
    
    \item \textbf{Architettura modulare estensibile}:
          \begin{itemize}
              \item Separazione netta business logic / UI
              \item Facile aggiunta di nuovi algoritmi
              \item Utility riusabili tra algoritmi
          \end{itemize}
    
    \item \textbf{Interfaccia moderna e accessibile}:
          \begin{itemize}
              \item Design responsive multi-dispositivo
              \item Feedback visivo in tempo reale
              \item Gestione errori user-friendly
          \end{itemize}
\end{enumerate}

\section{Limitazioni e Miglioramenti Futuri}

\subsection{Limitazioni Attuali}

\begin{itemize}
    \item \textbf{Formato immagine output}: Solo PNG per garantire lossless, limitando 
          compatibilità
    \item \textbf{Dimensioni file}: Immagini molto grandi (>10MB) possono causare timeout 
          su Streamlit Cloud
    \item \textbf{Assenza cifratura}: I dati non sono criptati, solo nascosti
    \item \textbf{Robustezza limitata}: Solo DWT resiste a compressione, nessun algoritmo 
          resiste a ritaglio significativo
\end{itemize}

\subsection{Sviluppi Futuri Proposti}

\subsubsection{Nuovi Algoritmi}

\begin{itemize}
    \item \textbf{DCT-based steganography}: Embedding nei coefficienti DCT (JPEG-friendly)
    \item \textbf{Spread Spectrum}: Distribuzione dati su frequenze multiple
    \item \textbf{Adaptive embedding}: Selezione automatica zone embedding in base a contenuto
\end{itemize}

\subsubsection{Funzionalità Aggiuntive}

\begin{enumerate}
    \item \textbf{Crittografia integrata}:
          \begin{lstlisting}
# Cifratura pre-embedding
from cryptography.fernet import Fernet

key = Fernet.generate_key()
cipher = Fernet(key)
encrypted_msg = cipher.encrypt(message.encode())

# Embedding del messaggio cifrato
hide_message(img, encrypted_msg)
          \end{lstlisting}
    
    \item \textbf{Stegoanalisi integrata}:
          \begin{itemize}
              \item Chi-square attack detection
              \item RS analysis
              \item Visual attack visualization
          \end{itemize}
    
    \item \textbf{Batch processing}:
          \begin{itemize}
              \item Upload multiplo immagini
              \item Distribuzione dati su più immagini
              \item Processing parallelo
          \end{itemize}
    
    \item \textbf{Supporto video}:
          \begin{itemize}
              \item Embedding in frame video
              \item Maggiore capacità totale
              \item Robustezza temporale
          \end{itemize}
    
    \item \textbf{API REST}:
          \begin{itemize}
              \item Integrazione programmatica
              \item Automazione workflows
              \item Servizio headless
          \end{itemize}
\end{enumerate}

\subsubsection{Ottimizzazioni Tecniche}

\begin{itemize}
    \item \textbf{GPU acceleration}: Uso CUDA per DWT su grandi immagini
    \item \textbf{Compressione adattiva}: Algoritmi di compressione context-aware per binari
    \item \textbf{Multi-threading}: Parallelizzazione embedding su canali RGB
    \item \textbf{Progressive encoding}: Streaming per file molto grandi
\end{itemize}

\section{Lezioni Apprese}

\subsection{Aspetti Tecnici}

\begin{itemize}
    \item \textbf{Trade-off inevitabili}: Impossibile ottimizzare simultaneamente capacità, 
          qualità e robustezza
    \item \textbf{Importanza validazione}: Controlli preventivi evitano computazioni 
          inutili e migliorano UX
    \item \textbf{Modularità cruciale}: Architettura modulare ha facilitato debugging 
          e estensioni
    \item \textbf{Testing essenziale}: Metriche quantitative fondamentali per valutare 
          algoritmi oggettivamente
\end{itemize}

\subsection{Gestione Progetto}

\begin{itemize}
    \item \textbf{Prototipazione rapida}: Streamlit ha permesso sviluppo UI molto veloce
    \item \textbf{CI/CD vantaggioso}: Automazione testing/deploy ha ridotto errori
    \item \textbf{Documentazione parallela}: Scrivere docs durante sviluppo facilita manutenzione
    \item \textbf{Version control disciplinato}: Commit atomici e branch feature hanno 
          organizzato lo sviluppo
\end{itemize}

\section{Considerazioni Finali}

Il progetto Steganography WebApp dimostra come tecniche teoriche di steganografia possano 
essere implementate in un'applicazione pratica e accessibile. L'architettura modulare e 
l'interfaccia intuitiva rendono il sistema utilizzabile sia per scopi didattici che 
applicativi reali.

L'implementazione di tre algoritmi con caratteristiche diverse offre un panorama completo 
delle possibilità della steganografia moderna, evidenziando i trade-off tra capacità, 
qualità e robustezza.

Il codice open-source e ben documentato può servire come base per ulteriori ricerche e 
sviluppi nel campo della steganografia digitale, facilitando l'integrazione di nuovi 
algoritmi e funzionalità.

\vspace{1cm}

\textit{Il progetto è disponibile pubblicamente su GitHub:}

\texttt{https://github.com/GiuseppeBellamacina/Steganography-WebApp}

\textit{Demo live su Streamlit Cloud:}

\texttt{https://steg-app.streamlit.app}

\newpage
