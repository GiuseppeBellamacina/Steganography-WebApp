\chapter{Conclusioni}

\section{Riepilogo del Progetto}

Il progetto Steganography WebApp ha raggiunto gli obiettivi prefissati, realizzando 
un'applicazione web completa e funzionale per la steganografia digitale~\cite{cox2007digital}. L'implementazione 
di tre algoritmi distinti (LSB, DWT e PVD) offre agli utenti la flessibilità di scegliere 
l'approccio più adatto alle loro esigenze specifiche.

\subsection{Obiettivi Raggiunti}

L'implementazione si è concretizzata in un sistema completo che integra tre algoritmi
steganografici funzionanti con caratteristiche complementari. Il supporto per formati multipli
(stringhe, immagini e file binari) amplia lo spettro applicativo, mentre l'interfaccia utente
intuitiva rende le operazioni accessibili anche a utenti non tecnici. Il sistema di metriche
integrato fornisce valutazioni quantitative della qualità, e l'automazione attraverso backup
parametri e preset ottimizzati semplifica notevolmente il flusso di lavoro. Il deployment su
Streamlit Cloud garantisce accessibilità da qualsiasi dispositivo senza necessità di
installazioni locali~\cite{streamlit2023}.

\section{Confronto Prestazioni}

\subsection{Risultati Sperimentali}

Test condotti su diverse immagini hanno confermato le aspettative teoriche:

\begin{table}[H]
\centering
\begin{tabular}{|l|c|c|c|}
\hline
\textbf{Metrica} & \textbf{LSB} & \textbf{DWT} & \textbf{PVD} \\ \hline
PSNR medio & 52.3 dB & 38.7 dB & 47.1 dB \\ \hline
SSIM medio & 0.9998 & 0.9945 & 0.9987 \\ \hline
Capacità (bpp) & 3.0 & 0.8 & 2.5 \\ \hline
Tempo medio* & 0.12s & 1.85s & 0.45s \\ \hline
\end{tabular}
\caption{Prestazioni medie su immagini 800×600 (*su CPU Intel i7)}
\end{table}

\subsection{Trade-off Osservati}

I test hanno confermato le caratteristiche attese di ogni algoritmo. LSB offre la migliore
qualità visiva e velocità, ma non resiste a compressione JPEG. DWT è l'unico robusto a
compressioni moderate, ma ha capacità ridotta e tempi più lunghi. PVD si posiziona come
compromesso intermedio, con buon bilanciamento capacità/qualità ma sensibile a
ridimensionamenti.

\section{Contributi Originali}

Rispetto alle implementazioni esistenti, il progetto introduce:

\begin{enumerate}
    \item \textbf{Header robusto multi-layer}:
          \begin{itemize}
              \item Magic header 64-bit per DWT (riduce false positive in rumore coefficienti)
              \item Magic header 16-bit per LSB/PVD (sufficiente in dominio spaziale)
              \item Lunghezza messaggio per lettura precisa
              \item Checksum per verifica integrità
              \item Terminatore per sicurezza
              \item \textbf{Estrazione two-phase per DWT}: prima legge header+size,
                    poi estrae esattamente il payload (previene lettura di coefficienti non modificati)
          \end{itemize}
    
    \item \textbf{Sistema preset intelligenti}:
          \begin{itemize}
              \item Configurazioni ottimizzate per casi d'uso comuni
              \item Parametri calibrati sperimentalmente
              \item Bilanciamento automatico capacità/qualità
          \end{itemize}
    
    \item \textbf{Architettura modulare estensibile}:
          \begin{itemize}
              \item Separazione netta business logic / UI
              \item Facile aggiunta di nuovi algoritmi
              \item Utility riusabili tra algoritmi
          \end{itemize}
    
    \item \textbf{Interfaccia moderna e accessibile}:
          \begin{itemize}
              \item Design responsive multi-dispositivo
              \item Feedback visivo in tempo reale
              \item Gestione errori user-friendly
          \end{itemize}
\end{enumerate}

\section{Limitazioni e Miglioramenti Futuri}

\subsection{Limitazioni Attuali}

Il sistema presenta alcune limitazioni legate principalmente a scelte implementative conservative.
L'output in formato PNG, pur garantendo qualità lossless indispensabile per il recupero corretto
dei dati, limita la compatibilità con sistemi che prediligono formati più compatti. Le dimensioni
delle immagini processabili su Streamlit Cloud sono vincolate dai timeout del servizio, rendendo
problematiche elaborazioni di file superiori a 10MB. L'assenza di cifratura integrata implica
che i dati siano solamente nascosti ma non protetti da accesso non autorizzato. La robustezza
rimane il tallone d'Achille: solo DWT resiste a compressioni moderate, mentre nessun algoritmo
sopravvive a ritagli significativi o ridimensionamenti sostanziali.

\subsection{Sviluppi Futuri Proposti}

Possibili migliormenti includono:

\begin{itemize}
    \item \textbf{Nuovi algoritmi}: DCT per compatibilità JPEG, spread spectrum per maggiore
          robustezza, embedding adattivo basato sul contenuto
    \item \textbf{Crittografia integrata}: Cifrare i dati prima dell'embedding per maggiore
          sicurezza
    \item \textbf{Stegoanalisi}: Strumenti per rilevare la presenza di dati nascosti (chi-square
          test, RS analysis)
    \item \textbf{Batch processing}: Elaborazione di più immagini simultaneamente
    \item \textbf{Supporto video}: Embedding in frame video per maggiore capacità
    \item \textbf{Ottimizzazioni}: GPU acceleration per immagini grandi, multi-threading per
          parallelizzare l'elaborazione
\end{itemize}

\section{Considerazioni Finali}

Il progetto Steganography WebApp dimostra come tecniche teoriche di steganografia possano 
essere implementate in un'applicazione pratica e accessibile. L'architettura modulare e 
l'interfaccia intuitiva rendono il sistema utilizzabile sia per scopi didattici che 
applicativi reali.

L'implementazione di tre algoritmi con caratteristiche diverse offre un panorama completo 
delle possibilità della steganografia moderna, evidenziando i trade-off tra capacità, 
qualità e robustezza.

Il codice open-source e ben documentato può servire come base per ulteriori ricerche e 
sviluppi nel campo della steganografia digitale, facilitando l'integrazione di nuovi 
algoritmi e funzionalità.

\vspace{1cm}

\textit{Il progetto è disponibile su:}

GitHub: \texttt{https://github.com/GiuseppeBellamacina/Steganography-WebApp}

Demo: \texttt{https://steg-app.streamlit.app}
