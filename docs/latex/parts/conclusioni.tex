\chapter{Conclusioni}

\section{Riepilogo del Progetto}

Il progetto Steganography WebApp ha raggiunto gli obiettivi prefissati, realizzando 
un'applicazione web completa e funzionale per la steganografia digitale~\cite{cox2007digital}. L'implementazione 
di tre algoritmi distinti (LSB, DWT e PVD) offre agli utenti la flessibilità di scegliere 
l'approccio più adatto alle loro esigenze specifiche.

\subsection{Obiettivi Raggiunti}

L'implementazione si è concretizzata in un sistema completo che integra tre algoritmi
steganografici funzionanti con caratteristiche complementari. Il supporto per formati multipli
(stringhe, immagini e file binari) amplia lo spettro applicativo, mentre l'interfaccia utente
intuitiva rende le operazioni accessibili anche a utenti non tecnici. Il sistema di metriche
integrato fornisce valutazioni quantitative della qualità, e l'automazione attraverso backup
parametri e preset ottimizzati semplifica notevolmente il flusso di lavoro. Il deployment su
Streamlit Cloud garantisce accessibilità da qualsiasi dispositivo senza necessità di
installazioni locali~\cite{streamlit2023}.

\section{Confronto Prestazioni}

\subsection{Risultati Sperimentali}

Test condotti su diverse immagini hanno confermato le aspettative teoriche:

\begin{table}[H]
\centering
\begin{tabular}{|l|c|c|c|}
\hline
\textbf{Metrica} & \textbf{LSB} & \textbf{DWT} & \textbf{PVD} \\ \hline
PSNR medio & 52.3 dB & 38.7 dB & 47.1 dB \\ \hline
SSIM medio & 0.9998 & 0.9945 & 0.9987 \\ \hline
Capacità (bpp) & 3.0 & 0.8 & 2.5 \\ \hline
Tempo medio* & 0.12s & 1.85s & 0.45s \\ \hline
\end{tabular}
\caption{Prestazioni medie su immagini 800×600 (*su CPU Intel i7)}
\end{table}

\subsection{Trade-off Osservati}

I test sperimentali hanno confermato le caratteristiche distintive di ciascun algoritmo.
LSB eccelle in qualità visiva e velocità di esecuzione, ma sacrifica completamente la
robustezza a compressione JPEG. DWT rappresenta l'unico approccio robusto a compressioni
JPEG fino al 90\%, ma questa resistenza comporta una capacità ridotta e costi computazionali
significativamente più elevati. PVD si posiziona come compromesso interessante, offrendo un
buon bilanciamento tra capacità e qualità con complessità intermedia, pur mostrando sensibilità
a operazioni di ridimensionamento.

\section{Contributi Originali}

Rispetto alle implementazioni esistenti, il progetto introduce:

\begin{enumerate}
    \item \textbf{Header robusto multi-layer}:
          \begin{itemize}
              \item Magic header 64-bit per DWT (riduce false positive in rumore coefficienti)
              \item Magic header 16-bit per LSB/PVD (sufficiente in dominio spaziale)
              \item Lunghezza messaggio per lettura precisa
              \item Checksum per verifica integrità
              \item Terminatore per sicurezza
              \item \textbf{Estrazione two-phase per DWT}: prima legge header+size,
                    poi estrae esattamente il payload (previene lettura di coefficienti non modificati)
          \end{itemize}
    
    \item \textbf{Sistema preset intelligenti}:
          \begin{itemize}
              \item Configurazioni ottimizzate per casi d'uso comuni
              \item Parametri calibrati sperimentalmente
              \item Bilanciamento automatico capacità/qualità
          \end{itemize}
    
    \item \textbf{Architettura modulare estensibile}:
          \begin{itemize}
              \item Separazione netta business logic / UI
              \item Facile aggiunta di nuovi algoritmi
              \item Utility riusabili tra algoritmi
          \end{itemize}
    
    \item \textbf{Interfaccia moderna e accessibile}:
          \begin{itemize}
              \item Design responsive multi-dispositivo
              \item Feedback visivo in tempo reale
              \item Gestione errori user-friendly
          \end{itemize}
\end{enumerate}

\section{Limitazioni e Miglioramenti Futuri}

\subsection{Limitazioni Attuali}

Il sistema presenta alcune limitazioni legate principalmente a scelte implementative conservative.
L'output in formato PNG, pur garantendo qualità lossless indispensabile per il recupero corretto
dei dati, limita la compatibilità con sistemi che prediligono formati più compatti. Le dimensioni
delle immagini processabili su Streamlit Cloud sono vincolate dai timeout del servizio, rendendo
problematiche elaborazioni di file superiori a 10MB. L'assenza di cifratura integrata implica
che i dati siano solamente nascosti ma non protetti da accesso non autorizzato. La robustezza
rimane il tallone d'Achille: solo DWT resiste a compressioni moderate, mentre nessun algoritmo
sopravvive a ritagli significativi o ridimensionamenti sostanziali.

\subsection{Sviluppi Futuri Proposti}

\subsubsection{Nuovi Algoritmi}

L'architettura modulare faciliterebbe l'integrazione di algoritmi aggiuntivi. Un'implementazione
basata su DCT (Discrete Cosine Transform) offrirebbe compatibilità nativa con JPEG, operando
direttamente sui coefficienti di compressione. Le tecniche spread spectrum distribuirebbero
i dati su multiple frequenze aumentando la robustezza al rumore. Approcci di embedding adattivo
potrebbero selezionare automaticamente le zone più adatte in base al contenuto dell'immagine,
migliorando il trade-off tra capacità e impercettibilità.

\subsubsection{Funzionalità Aggiuntive}

\begin{enumerate}
    \item \textbf{Crittografia integrata}:
          \begin{lstlisting}
# Cifratura pre-embedding
from cryptography.fernet import Fernet

key = Fernet.generate_key()
cipher = Fernet(key)
encrypted_msg = cipher.encrypt(message.encode())

# Embedding del messaggio cifrato
hide_message(img, encrypted_msg)
          \end{lstlisting}
    
    \item \textbf{Stegoanalisi integrata}:
          \begin{itemize}
              \item Chi-square attack detection
              \item RS analysis
              \item Visual attack visualization
          \end{itemize}
    
    \item \textbf{Batch processing}:
          \begin{itemize}
              \item Upload multiplo immagini
              \item Distribuzione dati su più immagini
              \item Processing parallelo
          \end{itemize}
    
    \item \textbf{Supporto video}:
          \begin{itemize}
              \item Embedding in frame video
              \item Maggiore capacità totale
              \item Robustezza temporale
          \end{itemize}
    
    \item \textbf{API REST}:
          \begin{itemize}
              \item Integrazione programmatica
              \item Automazione workflows
              \item Servizio headless
          \end{itemize}
\end{enumerate}

\subsubsection{Ottimizzazioni Tecniche}

\begin{itemize}
    \item \textbf{GPU acceleration}: Uso CUDA per DWT su grandi immagini
    \item \textbf{Compressione adattiva}: Algoritmi di compressione context-aware per binari
    \item \textbf{Multi-threading}: Parallelizzazione embedding su canali RGB
    \item \textbf{Progressive encoding}: Streaming per file molto grandi
\end{itemize}

\section{Lezioni Apprese}

L'esperienza di sviluppo ha evidenziato alcuni aspetti fondamentali. Dal punto di vista tecnico,
i trade-off tra capacità, qualità e robustezza si sono rivelati inevitabili: ogni tentativo di
ottimizzare una dimensione comporta sacrifici sulle altre. La validazione preventiva degli input
è emersa come pratica essenziale, evitando computazioni inutili e migliorando significativamente
l'esperienza utente. L'architettura modulare ha dimostrato il suo valore durante il debugging e
le estensioni successive, mentre le metriche quantitative si sono rivelate fondamentali per
valutazioni oggettive degli algoritmi.

Sulla gestione del progetto, Streamlit ha permesso una prototipazione rapida dell'interfaccia,
accelerando notevolmente lo sviluppo. L'automazione attraverso CI/CD ha ridotto gli errori in
produzione, e la documentazione parallela al codice ha facilitato la manutenzione. La disciplina
nel version control, con commit atomici e branch dedicati per le feature, ha mantenuto organizzato
il flusso di sviluppo.

\section{Considerazioni Finali}

Il progetto Steganography WebApp dimostra come tecniche teoriche di steganografia possano 
essere implementate in un'applicazione pratica e accessibile. L'architettura modulare e 
l'interfaccia intuitiva rendono il sistema utilizzabile sia per scopi didattici che 
applicativi reali.

L'implementazione di tre algoritmi con caratteristiche diverse offre un panorama completo 
delle possibilità della steganografia moderna, evidenziando i trade-off tra capacità, 
qualità e robustezza.

Il codice open-source e ben documentato può servire come base per ulteriori ricerche e 
sviluppi nel campo della steganografia digitale, facilitando l'integrazione di nuovi 
algoritmi e funzionalità.

\vspace{1cm}

\textit{Il progetto è disponibile pubblicamente su GitHub:}

\texttt{https://github.com/GiuseppeBellamacina/Steganography-WebApp}

\textit{Demo live su Streamlit Cloud:}

\texttt{https://steg-app.streamlit.app}
