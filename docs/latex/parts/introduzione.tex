\chapter{Introduzione}

\section{Contesto e Motivazioni}

La steganografia è l'arte e la scienza di nascondere informazioni all'interno di altri dati 
apparentemente innocui, in modo che la presenza stessa del messaggio nascosto sia difficile 
da rilevare~\cite{johnson1998exploring}. A differenza della crittografia, che rende il messaggio illeggibile ma non ne 
nasconde l'esistenza, la steganografia mira a mascherare completamente la comunicazione segreta.

Nel contesto moderno della sicurezza informatica e della privacy digitale, la steganografia 
trova applicazioni in diversi ambiti:

\begin{itemize}
    \item \textbf{Protezione del copyright}: Watermarking digitale per proteggere la proprietà 
          intellettuale di immagini, video e documenti
    \item \textbf{Comunicazioni sicure}: Trasmissione di informazioni sensibili senza destare 
          sospetti
    \item \textbf{Autenticazione}: Verifica dell'integrità e dell'autenticità di contenuti 
          multimediali
    \item \textbf{Privacy}: Protezione di dati personali in contesti dove la crittografia 
          potrebbe attirare attenzione indesiderata
\end{itemize}

\section{Obiettivi del Progetto}

Il presente progetto nasce con l'obiettivo di sviluppare un'applicazione web moderna e 
accessibile che implementi diversi algoritmi steganografici, permettendo agli utenti di:

\begin{enumerate}
    \item \textbf{Nascondere dati multipli}: Supportare testo, immagini e file binari 
          arbitrari
    \item \textbf{Scegliere l'algoritmo ottimale}: Offrire LSB, DWT e PVD con caratteristiche 
          complementari
    \item \textbf{Configurare i parametri}: Fornire preset ottimizzati e opzioni personalizzabili
    \item \textbf{Valutare la qualità}: Calcolare metriche quantitative (PSNR, MSE, SSIM) 
          per misurare l'impatto visivo
    \item \textbf{Recuperare dati facilmente}: Implementare un sistema di backup parametri 
          automatico
\end{enumerate}

\section{Struttura della Relazione}

La relazione è organizzata nei seguenti capitoli:

\begin{itemize}
    \item \textbf{Capitolo 1 - Fondamenti Teorici}: Introduce i concetti base della 
          steganografia e le tecniche tradizionali
    \item \textbf{Capitolo 2 - Algoritmi Implementati}: Descrive in dettaglio LSB, DWT e PVD
    \item \textbf{Capitolo 3 - Architettura del Sistema}: Presenta la struttura modulare 
          del progetto
    \item \textbf{Capitolo 4 - Implementazione}: Analizza gli aspetti tecnici e le scelte 
          implementative
    \item \textbf{Capitolo 5 - Interfaccia Utente}: Illustra il design e le funzionalità 
          dell'interfaccia web
    \item \textbf{Conclusioni}: Riassume i risultati e discute sviluppi futuri
\end{itemize}

\newpage
