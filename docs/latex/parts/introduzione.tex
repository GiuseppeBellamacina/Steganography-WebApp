\chapter*{Introduzione}
\markboth{\MakeUppercase{Introduzione}}{\MakeUppercase{Introduzione}}
\addcontentsline{toc}{chapter}{Introduzione}

\section*{Contesto e Motivazioni}

La steganografia è l'arte e la scienza di nascondere informazioni all'interno di altri dati 
apparentemente innocui, in modo che la presenza stessa del messaggio nascosto sia difficile 
da rilevare~\cite{johnson1998exploring}. A differenza della crittografia, che rende il messaggio illeggibile ma non ne 
nasconde l'esistenza, la steganografia mira a mascherare completamente la comunicazione segreta.

Nel contesto moderno della sicurezza informatica e della privacy digitale, la steganografia 
trova applicazioni in diversi ambiti:

\begin{itemize}
    \item \textbf{Protezione del copyright}: Watermarking digitale per proteggere la proprietà 
          intellettuale di immagini, video e documenti
    \item \textbf{Comunicazioni sicure}: Trasmissione di informazioni sensibili senza destare 
          sospetti
    \item \textbf{Autenticazione}: Verifica dell'integrità e dell'autenticità di contenuti 
          multimediali
    \item \textbf{Privacy}: Protezione di dati personali in contesti dove la crittografia 
          potrebbe attirare attenzione indesiderata
\end{itemize}

\section*{Obiettivi del Progetto}

Il progetto consiste nello sviluppo di un'applicazione web che implementa tre algoritmi
steganografici (LSB, DWT e PVD) per nascondere testo, immagini e file binari all'interno
di immagini. L'applicazione offre preset ottimizzati per semplificare l'uso e calcola
metriche di qualità (PSNR e SSIM) per valutare i risultati. Un sistema di backup automatico
facilita il recupero dei dati nascosti.
