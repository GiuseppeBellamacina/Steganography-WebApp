\chapter*{Abstract}
\addcontentsline{toc}{chapter}{Abstract}

Questo progetto presenta lo sviluppo di un'applicazione web completa per la steganografia, 
implementata utilizzando Python e Streamlit. L'applicazione offre tre algoritmi steganografici 
avanzati: LSB (Least Significant Bit), DWT (Discrete Wavelet Transform) e PVD (Pixel Value 
Differencing), permettendo agli utenti di nascondere e recuperare diversi tipi di dati 
(testo, immagini e file binari) all'interno di immagini.

L'architettura del progetto segue principi di clean code e modularità, separando la logica 
di business dall'interfaccia utente. Ogni algoritmo è implementato in modo indipendente, 
offrendo configurazioni personalizzabili e preset ottimizzati per bilanciare capacità, 
qualità e robustezza.

Il sistema include funzionalità avanzate come il calcolo di metriche di qualità (PSNR, 
SSIM), un sistema di backup automatico dei parametri e validazione completa degli input. 
L'interfaccia web, sviluppata con Streamlit, offre un'esperienza utente intuitiva con 
visualizzazione in tempo reale delle operazioni.
